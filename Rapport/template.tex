\documentclass{article}


\usepackage{arxiv}

\usepackage[utf8]{inputenc} % allow utf-8 input
\usepackage[T1]{fontenc}    % use 8-bit T1 fonts
\usepackage{hyperref}       % hyperlinks
\usepackage{url}            % simple URL typesetting
\usepackage{booktabs}       % professional-quality tables
\usepackage{amsfonts}       % blackboard math symbols
\usepackage{nicefrac}       % compact symbols for 1/2, etc.
\usepackage{microtype}      % microtypography
\usepackage{lipsum}

\title{Video Games Ontology}


\author{
  Alexandre Dazat \\
  Master's degree in Artificial Intelligence\\
  Telecom SudParis\\
  \texttt{alexandre.dazat@telecom-sudparis.eu} \\
  %% examples of more authors
   \And
  Julien Denize\\
  Master's degree in Artificial Intelligence\\
  Telecom SudParis\\
  \texttt{julien.denize@telecom-sudparis.eu} \\
}

\begin{document}
\maketitle

\begin{abstract}
Homework for November 15th and IA301. 
The following MIRO description is related to our Video Games Ontology. This document also contains the map of the classes involved and some relevant queries within the context of video games search. 
\end{abstract}


% keywords can be removed
\keywords{Ontology \and Video Games \and Protégé}

\section*{A Basics}

\subsection*{A.1 Ontology name}
Video-games ontology, v1.0 

\subsection*{A.2 Ontology owner}

Julien DENIZE, Alexandre DAZAT

\subsection*{A.3 Ontology license}

Creative Commons Attribution 3.0 (CC BY 3.0) 

\subsection*{A.4 Ontology URL}

URL 


\subsection*{A.5 Ontology repository }

URL
\subsection*{A.6 Methodological framework}


\section*{B. Motivation}


\subsubsection*{B.1 Need}

Single ontology resource that enables users to search games in databases with fine-grained granularity regarding the platforms, people involved in development, genre.

\subsection*{B.2 Competition}

The closest ontology to what we did can be found at the following URL : 
 
https://bartoc.org/en/node/18344

Besides, it mainly aims at encapsulating knowledge on events that happen in video games and information about players, whereas ours is made in order to store general information, as well as in depth features including genre or production team of games in order to retrieve it from large market databases.

\subsection*{B.3 Target audience}

The Video Game Ontology is used by video game digital distribution service platforms such as Steam or Amazon to provide quality modeling in games representation so that consumers can provide the most accurate queries and get to the product they are looking for or discover interesting ones based on the various functional attributes available. 

\section*{C. Scope, requirements, development community}
\subsection*{C.1 Scope and coverage}
The ontology focuses 
\subsection*{C.2 Development community}

Alexandre DAZAT & Julien DENIZE
\subsection*{C.3 Communication}

URL
\section*{D. Knowledge acquisition}

\subsection*{D.1 Knowledge acquisition methodology}

Personal general knowledge acquired through 15+ years of gaming and structured into an ontology through the coursework given in IA301. 
\subsection*{D.2 Source knowledge location}
We collected data from wikipedia and
\subsection*{D.3 Content selection}


In order to provide good quality representation the Video Games Ontology has to gather enough well-known games, production team leaders and games of diverse genre. The goal is to provide at least the means for retrieving the most commonly searched games, the more the database grows in individuals and the more consumers will find what they are looking for. 
\section*{E. Ontology content}

\subsection*{E.1 Knowledge Representation language}
OWL version 5.5. 
\subsection*{E.2 Development environment}
Protégé
\subsection*{E.3 Ontology metrics}
paste metrics 
\subsection*{E.4 Incorporation of other ontologies}
\subsection*{E.4 Incorporation of other ontologies}
\subsection*{E.6 Identifier generation policy}
\subsection*{E.7 Entity metadata policy}
\subsection*{E.8 Upper ontology}
\subsection*{E.9 Ontology relationships}
\subsection*{E.10 Axiom patterns }
\subsection*{E.11 Dereferenceable IRIs }
\section*{F. Managing Change}
\section*{F.1 Sustainability plan}
\subsection*{F.2 Entity deprecation strategy }
\subsection*{F.3 Versioning policy}
\section*{G. Quality Assurance}
\subsection*{G.1 Testing}
\subsection*{G.2 Evaluation}
\subsection*{G.3 Value of use}
\subsection*{G.4 Institutional endorsement}
\subsection*{G.5 Evidence of use}


\subsection{Headings: second level}
\lipsum[5]
\begin{equation}
\xi _{ij}(t)=P(x_{t}=i,x_{t+1}=j|y,v,w;\theta)= {\frac {\alpha _{i}(t)a^{w_t}_{ij}\beta _{j}(t+1)b^{v_{t+1}}_{j}(y_{t+1})}{\sum _{i=1}^{N} \sum _{j=1}^{N} \alpha _{i}(t)a^{w_t}_{ij}\beta _{j}(t+1)b^{v_{t+1}}_{j}(y_{t+1})}}
\end{equation}

\subsubsection{Headings: third level}
\lipsum[6]

\paragraph{Paragraph}
\lipsum[7]

\section{Examples of citations, figures, tables, references}
\label{sec:others}
\lipsum[8] \cite{kour2014real,kour2014fast} and see \cite{hadash2018estimate}.

The documentation for \verb+natbib+ may be found at
\begin{center}
  \url{http://mirrors.ctan.org/macros/latex/contrib/natbib/natnotes.pdf}
\end{center}
Of note is the command \verb+\citet+, which produces citations
appropriate for use in inline text.  For example,
\begin{verbatim}
   \citet{hasselmo} investigated\dots
\end{verbatim}
produces
\begin{quote}
  Hasselmo, et al.\ (1995) investigated\dots
\end{quote}

\begin{center}
  \url{https://www.ctan.org/pkg/booktabs}
\end{center}


\subsection{Figures}
\lipsum[10] 
See Figure \ref{fig:fig1}. Here is how you add footnotes. \footnote{Sample of the first footnote.}
\lipsum[11] 

\begin{figure}
  \centering
  \fbox{\rule[-.5cm]{4cm}{4cm} \rule[-.5cm]{4cm}{0cm}}
  \caption{Sample figure caption.}
  \label{fig:fig1}
\end{figure}

\subsection{Tables}
\lipsum[12]
See awesome Table~\ref{tab:table}.

\begin{table}
 \caption{Sample table title}
  \centering
  \begin{tabular}{lll}
    \toprule
    \multicolumn{2}{c}{Part}                   \\
    \cmidrule(r){1-2}
    Name     & Description     & Size ($\mu$m) \\
    \midrule
    Dendrite & Input terminal  & $\sim$100     \\
    Axon     & Output terminal & $\sim$10      \\
    Soma     & Cell body       & up to $10^6$  \\
    \bottomrule
  \end{tabular}
  \label{tab:table}
\end{table}

\subsection{Lists}
\begin{itemize}
\item Lorem ipsum dolor sit amet
\item consectetur adipiscing elit. 
\item Aliquam dignissim blandit est, in dictum tortor gravida eget. In ac rutrum magna.
\end{itemize}


\bibliographystyle{unsrt}  
%\bibliography{references}  %%% Remove comment to use the external .bib file (using bibtex).
%%% and comment out the ``thebibliography'' section.


%%% Comment out this section when you \bibliography{references} is enabled.
\begin{thebibliography}{1}

\bibitem{kour2014real}
George Kour and Raid Saabne.
\newblock Real-time segmentation of on-line handwritten arabic script.
\newblock In {\em Frontiers in Handwriting Recognition (ICFHR), 2014 14th
  International Conference on}, pages 417--422. IEEE, 2014.

\bibitem{kour2014fast}
George Kour and Raid Saabne.
\newblock Fast classification of handwritten on-line arabic characters.
\newblock In {\em Soft Computing and Pattern Recognition (SoCPaR), 2014 6th
  International Conference of}, pages 312--318. IEEE, 2014.

\bibitem{hadash2018estimate}
Guy Hadash, Einat Kermany, Boaz Carmeli, Ofer Lavi, George Kour, and Alon
  Jacovi.
\newblock Estimate and replace: A novel approach to integrating deep neural
  networks with existing applications.
\newblock {\em arXiv preprint arXiv:1804.09028}, 2018.

\end{thebibliography}


\end{document}
